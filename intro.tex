\section{Introduction}\label{sec:intro}

As a general rule, the properties of a fluid can be altered significantly from their bulk values when confined in tight spaces, with the magnitude of such alterations dependent on the characteristics of the confining walls, specifically, the affinity of the fluid to the surface\cite{Gelb_Phase_1999}.
Of particular scientific and technical interest is the effect of confinement on a fluid's phase behavior (or phase boundaries)\cite{Gelb_Phase_1999,Rouquerol_Adsorption_1999,Lowell_Characterization_2004}.
For example, fluid adsorption in porous materials, the focus of this work, serves as the basis for potential viable carbon capture technologies\cite{Khoo_Life_2006,Maginn_What_2010,Meek_Metal_2010,Zhou_Introduction_2012}.
While theory and simulation have played a historically significant role in the characterization of porous materials\cite{Gelb_Phase_1999,Gubbins_role_2011,Monson_Understanding_2012}, they have more recently been identified as key tools in screening and developing potential carbon capture materials\cite{Yang_Molecular_2007,Watanabe_Accelerating_2012,Lin_In_2012}.
Thus, the advancement of carbon capture technologies and other applications of gas adsorption will depend on the availability and further development of computationally efficient and precise methods to predict the thermodynamic and dynamic properties of fluids in porous materials.
At present, the two primary molecular modeling tools used in studies of adsorption are density functional theory\cite{Evans_Density_1992} (with which we include the closely related lattice mean field theories\cite{Oliveira_Lattice-gas_1978,Marconi_Microscopic_1989,Kierlik_Capillary_2001,Siderius_Predicting_2009}) and various forms of Monte Carlo (MC) molecular simulation\cite{Allen_Computer_1987,Frenkel_Understanding_1996}.
These methods have proven essential to the advancement of the fundamental understanding of adsorption processes\cite{Gubbins_role_2011} by, for example, suggesting the existence of cavitation-induced capillary evaporation in ink-bottle pores prior to its observance experimentally\cite{Sarkisov_Modeling_2001} and confirming the relationship between subcritical adsorption hysteresis and fluid metastability\cite{Neimark_Adsorption_2000,Neimark_Capillary_2001}.
