\section{Summary and Conclusions}\label{sec:summary}

In this work, we introduced a GC-TMMC method as an efficient technique for computing various properties of an adsorbed fluid for situations where multiple fluid phases, both stable and metastable, may be present.
The key statistical quantity exploited in this work is the PNPD, which is straightforwardly obtained from information about attempted transitions in an otherwise normal MC simulation.
The TMMC method selected here is that of Errington and co-workers, which is distinguished from earlier TMMC work by the introduction of a biasing function proportional to the logarithm of the aforementioned PNPD.
The self-correcting biasing function encourages the simulation to sample all states uniformly, allowing for satisfactory sampling of $N$-states with both low and high probability and the accurate construction of the entire PNPD, from which any number of ensemble averages can then be computed.

A particularly attractive feature of GC-TMMC is that a single simulation can provide the necessary information to generate various properties of the adsorptive system for any number of thermodynamic state points.
Thus, the PNPD from a single simulation is able to generate an entire adsorption isotherm, locate capillary phase transitions, identify the equilibrium phase transition, and compute other properties.
The example applications that we discussed in Sec 3 demonstrate that the GC-TMMC method generates adsorption isotherms that are consistent with both expected behavior of adsorbed fluids and with previously published simulation results generated by conventional GCMC and Gibbs ensemble simulations of fluid confined in adsorbing pores.
Additionally, we demonstrated the ability of GC-TMMC to compute other properties of adsorbed fluids by computing the isosteric heat of adsorption.
This property is more challenging to reliably calculate than simple ensemble averages as it is computed from ensemble fluctuations.
The $Q_{st}$ results given here for a family of Ar-MWCNT systems appear to contain less noise than calculations in existing GCMC simulations, which likely follows from the use of the PNPD to compute the necessary ensemble fluctuations.

A valid question that may be asked is whether the GC-TMMC method presents a computational advantage to conventional GCMC, since a particular GC-TMMC simulation is typically run for many more trial moves than an equivalent GCMC simulation.
While it is true that one GC-TMMC simulation might require more computational time than, say, the 20 or 50 conventional GCMC simulations used to generate an adsorption isotherm, it is important to keep in mind that a single GC-TMMC simulation provides access to the continuum of state points comprising the entire curve, which is the equivalent of hundreds or thousands of individual GCMC simulations.

The work here presents an initial application of the GC-TMMC method to adsorption problems involving hysteresis and we anticipate that this method can be widely exploited to improve modeling of adsorption and obtain properties that might be difficult to obtain with conventional or more established forms of MC simulation.
For example, if an adsorptive system could support more than two stable states, it is possible that existing MC methods might fail to identify some of those states.
In other applications, one can consider the use of temperature-expanded ensembles\cite{Lyubartsev_New_1991,Grzelak_Nanoscale_2010,Kumar_Monte_2011} and/or temperature reweighting\cite{Errington_Direct_2003} of a macrostate probability distribution to obtain additional temperature-dependent information from a GC-TMMC simulation.
This might be used to rigorously search for the hysteresis critical temperature or simply to compute adsorption isotherms and associated properties for more than one temperature, all from a single simulation.
Also, since GC-TMMC falls into the class of flat histogram simulation methods, it is possible to combine its algorithm with other flat histogram strategies to further improve its efficiency.
One such approach is that suggested by Shell et al.\cite{Shell_improved_2003} in which the Wang-Landau method\cite{Wang_Efficient_2001,Wang_Determining_2001,Shell_Generalization_2002,Shell_improved_2003} is used to quickly generate the biasing potential while using GC-TMMC to compute the PNPD.
Finally, though the present discussion involved fairly simple fluid models (single-site LJ Ar and rigid multisite CO\sub{2}), GC-TMMC is no less suited to more complicated fluid models and advanced MC sampling strategies, and modeling of adsorption of complex fluids in various media is just another application that can be considered.

